\documentclass[12pt, letterpaper]{article}
\usepackage[english,russian]{babel}
\usepackage[usenames]{color}

\title{\textbf{CodeStyle}}
\author{Седова Анна Алексеевна, группа ИА-032}

\date{Февраль 2022}

\begin{document}
\maketitle
\newpage
\section{Введение:}
Code Style используется для предоставления единых правил написания кода для конкретного проекта. Гораздо проще читать/писать/редактировать код, когда проект имеет единый стиль.
\\
\begin{center}
    \LARGE\textbf{Code Style C++}\hline
\end{center}

\begin{itemize}
    \item \Large\textbf{Пробелы и отступы}
\end{itemize}
\setcounter{section}{0}
\section{Переменные с нового абзаца}
В отступах используется tab (размер составляет 4 пробела)
\begin{flushleft}
\begin{tabular}{ |l| } 
 \hline
\textcolor{blue}{int} \textcolor{cyan}{x = 7;} \\
\textcolor{blue}{double} \textcolor{cyan}{y = 5.17;} \\
 \hline
\end{tabular}
\end{flushleft}
\section{Отделять пробелами фигурные скобки}
\begin{flushleft}
\begin{tabular}{ |l| } 
 \hline
\\
\textcolor{blue}{if} ( \textcolor{cyan}{x > y}  ) \{ \\
\qquad ... \\
\} \\ 
\\
 \hline
\end{tabular}
\end{flushleft}
\section{Пробелы между операторами и операндами}
\begin{flushleft}
\begin{tabular}{ |l| } 
 \hline
\textcolor{blue}{int} \textcolor{cyan}{x }= (a + b) * c / d; \\
 \hline
\end{tabular}
\end{flushleft}
\begin{itemize}
\vspace*{10mm}
\hline
    \item \Large\textbf{Названия и переменные}
\end{itemize}
\\-Дать переменным описательные имена;
\\-Избегать однобуквенных названий (пример. x или c).
\setcounter{section}{0}
\section{Текстовая строка}
\begin{flushleft}
\begin{tabular}{ |l| } 
 \hline
\textcolor{blue}{string} str = \textcolor{green}{"Hello, World!";} \\
\hline
\end{tabular}
\end{flushleft}
\section{Ссылаться на const, а не на её значение}
\begin{flushleft}
\begin{tabular}{ |l| } 
\hline
    \textcolor{blue}{const int} AGE = 20;\\
\hline
\end{tabular}
\end{flushleft}
\section{Класс}
-Класс следует объявлять в заголовочном файле (.h) и определять (реализовывать) в файле исходного кода (.cpp);\\
-Имена файлов совпадают с именем класса.\\
\begin{flushleft}
\begin{tabular}{ |l| } 
\hline
\textcolor{blue}{class MyClass }\{\\
\ \ \ \   ...\\
\}\\
\hline
\end{tabular}
\end{flushleft}
    \begin{itemize}
\vspace*{10mm}
\hline
    \item \Large\textbf{Базовые выражения}
\end{itemize}
\setcounter{section}{0}
\section{Вывод на консоль}
\begin{flushleft}
\begin{tabular}{ |l| } 
\hline
cout << "Hello, world!" << endl;\\
\hline
\end{tabular}
\end{flushleft}
\section{Цикл for}
Когда известно кол-во повторений\\
\begin{flushleft}
\begin{tabular}{ |l| } 
\hline
\textcolor{blue}{for }(int i = 0; i < 5; i++) \{\\
\ \ \ \ ...\\
\}\\
\hline
\end{tabular}
\end{flushleft}
\section{Цикл while}
Когда неизвестно кол-во повторений\\
\begin{flushleft}
\begin{tabular}{ |l| } 
\hline
string str;\\
\textcolor{green}{while } (input >> str) \{\\
\ \ \ \ ...\\
\}\\
Повторяет, пока больше не будет строк\\
\hline
\end{tabular}
\end{flushleft}

\section{if/else, for, while}
-Использовать \{\} и соответствующие отступы\\
\begin{flushleft}
\begin{tabular}{ |l| } 
\hline
\textcolor{green}{if }(size == 0) \{\\
\ \ \ \ \textcolor{green}{return;}\\
\} \textcolor{green}{else } \{\\
\ \ \ \ \textcolor{green}{for }(int i = 0; i < 7; i++) \{\\
\ \ \ \ ...\\
\ \ \ \ \}\\
\}\\
\hline
\end{tabular}
\end{flushleft}
\section{Логический тип}
Не проверять значения логического типа, используя == или != с true или false\\
\begin{flushleft}
\begin{tabular}{ |l| } 
\hline
Плохая практика\\
\textcolor{green}{if }(x == true) \{\\
\ \ \ \ ...\\
\} \textcolor{green}{else if }(x != true) \{\\
\ \ \ \ ...\\
\}\\
\\
Хорошая практика\\
\textcolor{green}{if }(x) \{\\
\ \ \ \ ...\\
\} \textcolor{green}{else }\{\\
\ \ \ \ ...\\
\}\\
\hline
\end{tabular}
\end{flushleft}

\section{Имена функций}
\\-Имя функции начинается с прописной буквы и каждое слово в имени пишется с прописной буквы;
\\-Закрывающая скобка ставится ровно под своей функцией.
\begin{flushleft}
\begin{tabular}{ |l| } 
\hline
\textcolor{red}{void} \textcolor{blue}{AddTableEntry} ();\\
\textcolor{red}{void} \textcolor{blue}{DeleteUrl} ();\\
\textcolor{red}{void} \textcolor{blue}{OpenFileOrDie} ();\\
\hline
\end{tabular}
\end{flushleft}
\begin{itemize}
\newpage
\item\Large\textbf{Вывод}
\end{itemize}
Благодаря данной работе, я освоила команды LaTex и выполнила отчёт по основным правилам CodeStyle С++. 
\begin{thebibliography}{3}
\bibitem{1} 
ISO/IEC 14882 Programming languages — C++.
\bibitem{2} 
Руководство Google по стилю в C++ : [electronic resource].
URL : https://habr.com/ru/post/477722/ (Дата обращения: 09.02.2022)
\bibitem{3}
Осваиваем LaTeX за 30 минут : [electronic resource].
URL : https://habr.com/ru/company/ruvds/blog/574352/ (Дата обращения: 09.02.2022)
\end{thebibliography}
\end{document}
