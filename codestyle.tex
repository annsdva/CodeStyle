\documentclass[12pt, letterpaper]{article}
\usepackage[english,russian]{babel}
\usepackage[usenames]{color}
\usepackage{minted}

\title{\textbf{CodeStyle}}
\author{Седова Анна Алексеевна, группа ИА-032}
\date{Февраль 2022}

\begin{document}
\maketitle
\newpage
\section{Введение:}
Code Style используется для предоставления единых правил написания кода для конкретного проекта. Гораздо проще читать/писать/редактировать код, когда проект имеет единый стиль.
\section{Язык: С++}

\newpage
\begin{center}
    \LARGE\textbf{Code Style C++}\hline
\end{center}

\begin{itemize}
    \item \Large\textbf{Пробелы и отступы}
\end{itemize}
\setcounter{section}{0}
\section{Переменные с нового абзаца}
\begin{flushleft}
\begin{tabular}{ |l| } 
 \hline
\textcolor{blue}{int} \textcolor{cyan}{x = 7;} \\
\textcolor{blue}{double} \textcolor{cyan}{y = 5.17;} \\
 \hline
\end{tabular}
\end{flushleft}
\section{Отделять пробелами фигурные скобки}
\begin{flushleft}
\begin{tabular}{ |l| } 
 \hline
\\
\textcolor{blue}{if} ( \textcolor{cyan}{x > y}  ) \{ \\
\qquad ... \\
\} \\ 
\\
 \hline
\end{tabular}
\end{flushleft}
\section{Пробелы между операторами и операндами}
\begin{flushleft}
\begin{tabular}{ |l| } 
 \hline
\textcolor{blue}{int} \textcolor{cyan}{x }= (a + b) * c / d; \\
 \hline
\end{tabular}
\end{flushleft}
\begin{itemize}
\vspace*{10mm}
\hline
    \item \Large\textbf{Названия и переменные}
\end{itemize}
\\-Дать переменным описательные имена;
\\-Избегать однобуквенных названий (пример. x или c).
\setcounter{section}{0}
\section{Текстовая строка}
\textcolor{blue}{string} str = \textcolor{green}{"Hello, World!";} \\
\section{Ссылаться на const, а не на её значение}
    \textcolor{blue}{const int} AGE = 20;
\section{Класс}
\begin{minted}[fontsize=\footnotesize]{c++}
class MyClass {
    ...
}
\end{minted}

    \begin{itemize}
\vspace*{10mm}
\hline
    \item \Large\textbf{Базовые выражения}
\end{itemize}
\setcounter{section}{0}
\section{Вывод на консоль}
\begin{minted}[fontsize=\footnotesize]{c++}
cout << "Hello, world!" << endl;
\end{minted}
\section{Цикл for (когда известно кол-во повторений)}
\begin{minted}[fontsize=\footnotesize]{c++}
for (int i = 0; i < 5; i++) {
    ...
}
\end{minted}
\section{Цикл while (когда неизвестно кол-во повторений)}
\begin{minted}[fontsize=\footnotesize]{c++}
// Повторяет, пока больше не будет строк
string str;
while (input >> str) {
    ...
}
\end{minted}
\vspace*{20mm}
\section{if/else, for, while - использовать {} и соответствующие отступы}
\begin{minted}[fontsize=\footnotesize]{c++}
if (size == 0) {
    return;
} else {
    for (int i = 0; i < 7; i++) {
        ...
    }
}
\end{minted}
\section{Не проверять значения логического типа, используя == или != с true или false}
\begin{minted}[fontsize=\footnotesize]{c++}
// Плохая практика
if (x == true) {
    ...
} else if (x != true) {
    ...
}

// Хорошая практика
if (x) {
    ...
} else {
    ...
}
\end{minted}
\section{Имена функций}
\\-Имя функции начинается с прописной буквы и каждое слово в имени пишется с прописной буквы
\begin{minted}[fontsize=\footnotesize]{c++}
void AddTableEntry();
void DeleteUrl();
void OpenFileOrDie();
\end{minted}
\newpage
\begin{itemize}
\item\Large\textbf{Вывод}
\end{itemize}
Благодаря данной работе, я освоила команды LaTex и выполнила отчёт по основным правилам CodeStyle С++. 
\begin{thebibliography}{3}
    \bibitem{1} https://habr.com/ru/company/ruvds/blog/574352/
    \bibitem{2} http://dkhramov.dp.ua/Comp.RedefiningMaketitle#.Yf6FBbpBxPY
    \bibitem{3} http://mydebianblog.blogspot.com/2012/12/latex.html?showComment=1355584281523
\end{thebibliography}
\end{document}
